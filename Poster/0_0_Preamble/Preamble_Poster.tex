% !TeX program = pdflatex
% !TeX TXS-program:compile = txs:///pdflatex/
% !TeX TS-program = pdflatex
% !BIB program = biber
% !TeX TXS-program:bibliography = txs:///biber




%%%%%%%%%%%%%%%%%%%%%%%%%%%%
%%  FUNDAMENTAL PACKAGES  %%
%%%%%%%%%%%%%%%%%%%%%%%%%%%%


\usepackage[utf8]{inputenc}

\usepackage[LGR, T1]{fontenc}  % LGR needed for sansserif math

\usepackage[ngerman, american, USenglish, french]{babel}  % German and US English hyphenation and quotation marks
\selectlanguage{french}

\usepackage{calc}  % Enables calculations for lengths
\usepackage[nomessages]{fp}	 % Enables calculations in LaTeX

\usepackage{etoolbox}  % Enables manipulating LaTeX commmands via \preto, \appto, \patchcmd, etc.
\usepackage{xpatch}  % Enables manipulating LaTeX commmands via \xpatchcmd etc.

\PassOptionsToPackage{cmyk}{xcolor}
\definecolor{UBonnBlue}  {cmyk}{1.00, 0.70, 0.00, 0.00}
\definecolor{UBonnYellow}{cmyk}{0.00, 0.20, 1.00, 0.00}
\definecolor{UBonnGray}  {cmyk}{0.00, 0.00, 0.15, 0.55}
\colorlet{SpotColor}{UBonnBlue}
\definecolor{CMYKBlack}{cmyk}{0.00, 0.00, 0.00, 1.00}
\colorlet{CMYKGray}{CMYKBlack!50!}
\setbeamercolor{normal text}{fg = CMYKBlack}
\setbeamercolor{alerted text}{fg = SpotColor}
\setbeamerfont{alerted text}{family = \sffamily, series = \bfseries}

\setbeamercolor{block alerted title}{bg = SpotColor, fg = white}
\setbeamercolor{block body}{fg = CMYKBlack}

\setbeamercolor{title in headline}{fg = SpotColor}
\setbeamercolor{institute in headline}{fg = CMYKBlack}

\hypersetup{
	colorlinks = true,
	allcolors = SpotColor
}
\urlstyle{same}
\let \urlorig \url
\renewcommand{\url}[1]{\textcolor{SpotColor}{\urlorig{#1}}}
\let \hreforig \href
\renewcommand{\href}[2]{\textcolor{SpotColor}{\hreforig{#1}{#2}}}
\newcommand{\email}[1]{\href{mailto:#1}{\nolinkurl{#1}}}

\usepackage[absolute,overlay]{textpos}
\setlength{\TPboxrulesize}{1pt}
\textblockrulecolor{SpotColor}

\usepackage{ragged2e}

% microtype enables ``hanging punctuation'' and other neat microtypographic features
\ifxetex
	\usepackage[protrusion = true, expansion = false]{microtype}
\else
	\usepackage[protrusion = true, expansion = false, kerning = true]{microtype}
\fi

% Use the bm (= boldmath) package for better support of setting math in bold ==>
% Prevent the "Too many math fonts used" error:
\newcommand{\bmmax}{0}
\newcommand{\hmmax}{0}
\usepackage{bm}
% <==

% Packages for creating better-looking tables
\usepackage{booktabs}
\setlength{\cmidrulewidth}{0.67pt}
\setlength{\lightrulewidth}{0.67pt}
\setlength{\heavyrulewidth}{1pt}
\addtolength{\abovetopsep}{25pt}
\addtolength{\aboverulesep}{5pt}	% Make tables a little more spacious
\addtolength{\belowrulesep}{7.5pt}
\addtolength{\belowbottomsep}{20pt}
\usepackage{colortbl} % e.g., for colored rules
\arrayrulecolor{SpotColor} % Color all table rules blue

\usepackage{multirow}
\usepackage{tabularx}
\newcolumntype{C}{>{\centering\arraybackslash}X}
\newcolumntype{J}{>{\arraybackslash}X}
\newcolumntype{L}{>{\raggedright\arraybackslash}X}
\newcolumntype{R}{>{\raggedleft\arraybackslash}X}
\usepackage{makecell}
\usepackage{siunitx}[=v2]	% Allow, among others, for alignment of decimal numbers in tables at the decimal point.
\sisetup{
	detect-all,
	round-integer-to-decimal = true,
	tight-spacing            = true,
	group-digits             = true,
	group-minimum-digits     = 5,
	group-separator          = {\kern 1pt},
	table-align-text-pre     = false,
	table-align-text-post    = false,
	input-signs              = + -,
	input-symbols            = {*} {**} {***},
	input-open-uncertainty   = ,
	input-close-uncertainty  = ,
	table-align-text-post    = false,
	retain-explicit-plus
}
\newcolumntype{T}[1]{%
	S[table-space-text-pre = {\;\;\;\:}, table-space-text-post = {\;\;\;\:}, table-format = #1]%
}
\newcommand{\minitab}[2][l]{%
	\begin{tabular}{@{} #1 @{}}#2\end{tabular}%
}
\newcommand{\tworowlabel}[2]{%
	\multirow{2}{*}{\minitab{#1\\\quad{}#2}}%
}




%%%%%%%%%%%%%%%%%%%%%%%
%%  BEAMER SETTINGS  %%
%%%%%%%%%%%%%%%%%%%%%%%


% Use the beamerposter package for laying out the poster:
% \usepackage[size = a0, scale = 1.1]{beamerposter}
	% A0: 840 mm by 1188 mm in beamerposter
\usepackage[size = a1, scale = 1.2, orientation=portrait]{beamerposter}
	% 841 mm by 1189 mm: A0
\beamertemplatenavigationsymbolsempty

\usefonttheme{professionalfonts}

\newcommand{\highlight}[1]{\alert{\textbf{#1}}}

% Make itemize, enumerate, and description justified instead of flushleft
% ==>
\let\tempItemizeBeg\itemize
\let\tempItemizeEnd\enditemize
\renewenvironment{itemize}{\tempItemizeBeg\justifying}{\tempItemizeEnd}
\let\tempEnumerateBeg\enumerate
\let\tempEnumerateEnd\endenumerate
\renewenvironment{enumerate}{\tempEnumerateBeg\justifying}{\tempEnumerateEnd}
\let\tempDescriptionBeg\description
\let\tempDescriptionEnd\enddescription
\renewenvironment{description}{\tempDescriptionBeg\justifying}{\tempDescriptionEnd}
% <==

% Add white space before lists and between items inside lists
% ==>
\makeatletter
\def\@listi{\leftmargin\leftmarginii
	\topsep 0.333333\baselineskip % Spacing before lists
	\parsep 0\p@ \@plus\p@
	\itemsep 0.333333\baselineskip} % Spacing between items
\makeatother
% <==
\setbeamersize{description width = 0.55\leftmarginii}

\setbeamertemplate{enumerate item}{\textbf{\theenumi}}
\setbeamertemplate{itemize item}{\raisebox{-1.5pt}{\textbf{\textbullet}}}
\setbeamercolor{item}{fg = SpotColor}

\setbeamerfont{structure}{family = \sffamily, shape = \upshape, series = \bfseries}

\setbeamerfont{title}{series = \bfseries}
\setbeamercolor{title}{fg = SpotColor}

% Poster title
\setbeamertemplate{headline}{}
\setbeamertemplate{footline}{}
\AtBeginDocument{%
	\begin{textblock*}{\paperwidth-\leftmargin-\rightmargin}(\leftmargin, \topmargin)
		\vspace{-12.5pt}\renewcommand{\baselinestretch}{0.85}%
		\Huge\textbf{\textsf{\textcolor{SpotColor}{%
			\strut\inserttitle\strut%
		}}}\\[-40pt]
		{\color{SpotColor}\rule{\linewidth}{1pt}\\[-13.5pt]}
		\usebeamercolor{author in headline}{\Large\textsf{\strut\insertauthor\strut}}\\[-17.5pt]
		\usebeamercolor{institute in headline}{\large\textsf{\strut\insertinstitute\strut}}
	\end{textblock*}%
}
% Regular block definition
\setbeamerfont{block title}{family = \sffamily, series = \bfseries}
\setbeamercolor{block title}{fg = SpotColor}
\setbeamertemplate{block begin}{
	\par\vskip\medskipamount
	\begin{beamercolorbox}[colsep* = 0ex, dp = 30pt, left]{block title}
		\vskip-0.25cm
		\usebeamerfont{block title}%
		\large\strut\insertblocktitle\\[-20pt]
		\rule{\linewidth}{1pt}\\[15pt]
	\end{beamercolorbox}
	{\parskip0pt\par}
	\ifbeamercolorempty[bg]{block title}
	{}
	{\ifbeamercolorempty[bg]{block body}{}{\nointerlineskip\vskip-0.5pt}}
	\usebeamerfont{block body}
	\vskip-0.75cm
	\begin{beamercolorbox}[colsep* = 0ex,vmode]{block body}
		\setlength{\baselineskip}{36pt}%
		\sloppypar\justifying%
}
\setbeamertemplate{block end}{
	\end{beamercolorbox}
	\vskip\smallskipamount
}

% \left(  block definition (with frame)
\setbeamerfont{block alerted title}{parent = structure, series = \fontseries{\savesfbfseries}\selectfont}
\newlength{\inboxwd}
\newlength{\iinboxwd}
\newlength{\inboxrule}
\setbeamertemplate{block alerted begin}{%
	\par
	\begin{beamercolorbox}[sep = 25pt, rounded = false, dp = 2pt, leftskip = 12pt, rightskip = 12pt]{block alerted title}
		\colorlet{codecolor}{SpotColor!25}%
		\colorlet{SpotColor}{white}%
		\vskip-2pt%
		\usebeamerfont{block alerted title}%
		{\sloppy\large\insertblocktitle}%
	\end{beamercolorbox}
	{\parskip0pt\par}
	\usebeamerfont{block body}
	\vskip-0.25in
	\begin{beamercolorbox}[sep = 0.5cm, rounded = false, center]{block body}
	\setlength{\inboxwd}{\linewidth}
	\addtolength{\inboxwd}{-1cm}
		\begin{beamercolorbox}[rounded = false, wd = {\inboxwd}, center]{block body}
		\setlength{\iinboxwd}{\inboxwd}
		\setlength{\inboxrule}{\inboxwd}
		\addtolength{\iinboxwd}{-0.5cm}
		\addtolength{\inboxrule}{0.5cm}
		\begin{center}
			\begin{minipage}{\iinboxwd}
				\centering%
				\sffamily%
}
\setbeamertemplate{block alerted end}{%
				\end{minipage}
			\end{center}
		\end{beamercolorbox}
	\end{beamercolorbox}
	\vskip\smallskipamount%
}




%%%%%%%%%%%%%%%%%%%%%%%%%%%%%
%%  BIBLIOGRAPHY SETTINGS  %%
%%%%%%%%%%%%%%%%%%%%%%%%%%%%%


%\usepackage{natbib}
\newlength{\bibindent}
\newlength{\bibitemsep}
\newlength{\bibsep}

\setbeamertemplate{bibliography item}{}
\setbeamercolor{bibliography entry author}  {fg = SpotColor}
\setbeamercolor{bibliography entry title}   {fg = SpotColor}
\setbeamercolor{bibliography entry location}{fg = SpotColor}
\setbeamercolor{bibliography entry note}    {fg = SpotColor}




%%%%%%%%%%%%%%%%%%%%%
%%  MISCELLANEOUS  %%
%%%%%%%%%%%%%%%%%%%%%


\newcommand{\affilindicator}[1]{\textsuperscript{\kern.75pt \textit{#1}}}
\usepackage[math]{blindtext}  % For debugging only
\makeatletter
\def\blindtext@american{}
\makeatother

% Allow for fine-grained scaling of font sizes
% ==>
\usepackage{relsize}
\renewcommand\RSpercentTolerance{1}
% Enabling slightly reduced font for CAPS:
\newcommand{\caps}[1]{\textscale{0.96}{\textls[35]{\MakeUppercase{#1}}}}
% <==

\colorlet{codecolor}{SpotColor!75}
\newcommand{\code}[1]{\textcolor{codecolor}{\texttt{#1}}}

\newcommand{\sidenote}[1]{{\color{gray}#1}}

% Miscellaneous additional mathematical symbols etc. (e.g., \coloneqq)
\usepackage{mathtools}
\newcommand{\divslash}{{{\kern-0.15em}\mathbin{/}{\kern-0.2em}}}




%%%%%%%%%%%%%%%%%%%%%%%
%%  DOCUMENT LAYOUT  %%
%%%%%%%%%%%%%%%%%%%%%%%


\setlength{\leftmargin}{40.5mm}
\setlength{\rightmargin}{\leftmargin}

\newenvironment{parblock}[1]{%
	\begin{block}{#1}%
		\setlength{\parindent}{\baselinestretch\baselineskip}%
		\noindent\strut\ignorespaces%
}%
{%
	\end{block}%
}

\newcommand{\numcols}{3}
\FPeval{\numcolsminusone}{\numcols-1}

\newlength{\colsep}
\setlength{\colsep}{15.5mm}
\newlength{\totalcolsep}
\setlength{\totalcolsep}{\numcolsminusone\colsep}

\newlength{\colwidth}
\setlength{\colwidth}{(\paperwidth-\leftmargin-\rightmargin-\totalcolsep)/\numcols}

\setbeamersize{text margin left = \leftmargin, text margin right = \rightmargin}

\setlength{\topmargin}{34.5mm}
\setlength{\textheight}{\paperheight-2\topmargin}

\setlength{\fboxsep}{0pt}%
\setlength{\fboxrule}{1pt}%

\frenchspacing

\usepackage{tikz}
\usetikzlibrary{calc}
\usepackage{printlen}
\uselengthunit{mm}

\newcommand{\displaymargins}{%
	\begin{tikzpicture}[remember picture, overlay, x = 1mm, y = 1mm]
		\node[orange, anchor=west] at ($(current page.north west) + (\leftmargin, -\topmargin)$) {\textsf{\textbf{\printlength{\leftmargin}, \printlength{\topmargin}}}};
		% Top margin
		\draw[orange] ($(current page.north west) + (\leftmargin, -\topmargin)$) -- ($(current page.north east) + (-\rightmargin, -\topmargin)$);
		% Left margin
		\draw[orange] ($(current page.north west) + (\leftmargin, -\topmargin)$) -- ($(current page.south west) + (\leftmargin, \topmargin)$);
		% Right margin
		\draw[orange] ($(current page.north east) + (-\rightmargin, -\topmargin)$) -- ($(current page.south east) + (-\rightmargin, \topmargin)$);
		% Bottom margin
		\draw[orange] ($(current page.south west) + (\leftmargin, \topmargin)$) -- ($(current page.south east) + (-\rightmargin, \topmargin)$);
	\end{tikzpicture}%
}
\newcommand{\displaygrid}{%
	\begin{tikzpicture}[remember picture, overlay, x = 1mm, y = 1mm]
		% Horizontal rules
		\foreach \i in {0, 20, 40, 60, ..., 1120}{
			\draw[magenta!50] ($(current page.north west) + (\leftmargin, -\topmargin) + (0, -\i)$) -- ($(current page.north east) + (-\rightmargin, -\topmargin) + (0, -\i)$);
		}
		% Vertical rules
		\foreach \i in {0, 20, 40, 60, ..., 760}{
			\draw[magenta!50] ($(current page.north west) + (\leftmargin, -\topmargin) + (\i, 0)$) -- ($(current page.south west) + (\leftmargin, \topmargin) + (\i, 0)$);
		}
	\end{tikzpicture}%
}