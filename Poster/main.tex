% !TeX program = pdflatex
% !TeX TS-program = pdflatex
% !TeX TXS-program:compile = txs:///pdflatex/

% !BIB program = biber
% !BIB TS-program = biber
% !TeX TXS-program:bibliography = txs:///biber




\documentclass{beamer}

\usepackage{csquotes}
\usepackage{import}
\input    {0_0_Preamble/Preamble_Poster}
\input    {0_0_Preamble/Preamble_BibLaTeX_AER}  %\input{0_0_Preamble/Preamble_BibLaTeX_APA}
\subimport{0_0_Preamble/}{Preamble_Fonts_Charter_FiraSans}
\input    {0_0_Preamble/Preamble_Sansserif_math}

% Set up desired deviations from the default bibliography settings here:
\ExecuteBibliographyOptions{
	backref = false, giveninits = true,
	uniquename = false, uniquelist = false
}
\AtEveryBibitem{\clearfield{doi}}
\AtEveryBibitem{\clearfield{url}}
\AtEveryBibitem{\clearfield{eprint}}

\bibliography{Library}




%%%%%%%%%%%%%%%%%%%%%%%%%%%%%%%%%%%%%%%%%%%%%%%%%%%
%%  DOCUMENT-SPECIFIC ADDITIONS TO THE PREAMBLE  %%
%%%%%%%%%%%%%%%%%%%%%%%%%%%%%%%%%%%%%%%%%%%%%%%%%%%


\newlength{\blockOne}
\newlength{\blockTwo}
\newlength{\blockTwoandHalf}
\newlength{\blockThree}
\newlength{\blockFour}

% Vertical positions
% \setlength{\blockOne}  {12.75cm}
% \setlength{\blockTwo}  {44.00cm}
% \setlength{\blockThree}{71.00cm}
% \setlength{\blockFour} {99.50cm}
\setlength{\blockOne}  {.107\paperheight}
\setlength{\blockTwo}  {.37\paperheight}
\setlength{\blockTwoandHalf}  {.53\paperheight}
\setlength{\blockThree}{.60\paperheight}
\setlength{\blockFour} {.74\paperheight}

\title{La figure de l'économiste programmeur : histoire et perspectives}
\author{%
	Ambroise Warnery, \email{ambroise.warnery@gmail.com}
}
\institute{%
Université Paris 1 Panthéon Sorbonne, France
} % Institution(s)

\date{20/05/2025}

\usepackage{xspace}
\newcommand{\balA}[1][1]{BAL$^\mathup{I}_{#1:#1}$\xspace}
\newcommand{\unbalA}[1][n]{UNBAL$^\mathup{I}_{1:#1}$\xspace}
\newcommand{\balB}[1][1]{BAL$^\mathup{II}_{#1:#1}$\xspace}
\newcommand{\unbalB}[1][n]{UNBAL$^\mathup{II}_{#1:1}$\xspace}

\newcommand{\mynum}[1]{\scalebox{1.075}{\firasemibold#1}}




%%%%%%%%%%%%%%%%%%%%%
%%  DOCUMENT BODY  %%
%%%%%%%%%%%%%%%%%%%%%


\begin{document}


%\displaygrid
%\displaymargins


\begin{frame}[t]	% Entire poster is one large frame


%%%%%%%%%%%%%%%%%%%%%%%%%%%%%  FIGURES  %%%%%%%%%%%%%%%%%%%%%%%%%%%%%%


% Recommendation:
% Place figures and tables at the top of the poster.
% This way they will be at eye level.
% You will mostly be talking about your figures and tables, while
% the body text is largely supplementary and can thus be placed
% in the bottom half of your poster (at hip level ...).

% Surround figures by frame
\TPshowboxestrue


%%%%%%%%%%%%%%%%%%%%%%%%%%%%%%  Introduction  %%%%%%%%%%%%%%%%%%%%%%%%%%%%%%%


\begin{textblock*}{\colwidth}(\leftmargin, \blockOne)

	\begin{alertblock}{1 - Introduction.}
	\RaggedRight
	\vspace{0.25cm}

	Depuis sa création, l'informatique s'est progressivement imposée comme un outil incontournable et a profondément transformé la recherche en économie (\cite{backhouseItsComputersStupid2017}). Si cette mutation technique est souvent analysée à travers l’évolution des méthodes ou des outils, elle repose aussi sur des trajectoires individuelles : celles d’économistes-programmeurs, à la fois chercheurs et développeurs, dont les logiciels sont devenus structurants dans plusieurs sous-champs de la discipline. La pratique de la programmation, et ses auteurs ont peu été étudiés, en raison d'une faible visibilité dans les travaux universitaires, et d'un manque de sources.
	\vspace{0.5em}

	Ce mémoire propose de documenter et d’analyser ces figures hybrides en s’appuyant sur une série d’entretiens issus du projet \textit{Computerization of Economics: Oral Histories}. Il s’agit de comprendre comment des économistes en viennent à créer des outils informatiques spécifiques, comment ces pratiques se diffusent, et quels enjeux épistémiques, institutionnels et professionnels elles soulèvent. En étudiant leurs parcours, ce travail éclaire une dimension encore peu reconnue du travail scientifique : le développement logiciel comme contribution à part entière à la production de savoirs en économie.
	\vspace{0.25cm}
	\end{alertblock}

\end{textblock*}



%%%%%%%%%%%%%%%%%%%%%%%%%%%%%%  Questions de Recherches  %%%%%%%%%%%%%%%%%%%%%%%%%%%%%%%


\begin{textblock*}{2\colwidth + \colsep}(\leftmargin, \blockTwoandHalf)

	\begin{alertblock}{2 - Questions de Recherches.}
	\RaggedRight
	\vspace{0.25cm}

	Dans un contexte où l’informatique structure de plus en plus les pratiques économiques, la figure de l’économiste-programmeur reste peu visible et rarement étudiée. Pourtant, ces chercheurs hybrides — à la fois producteurs de savoirs et développeurs d’outils — ont joué un rôle crucial dans la transformation des méthodes économiques contemporaines.

	Ce mémoire s’articule autour de trois questions centrales :
		  \begin{itemize}
			\item Qui sont les économistes-programmeurs, et quelles sont les conditions sociales, techniques et institutionnelles qui rendent possible leur émergence ?
			\item Comment naissent les projets de logiciels ? comment se développent et diffusent-ils ?
			\item Comment l'identité professionnelle des « économistes-programmeurs » a-t-elle évolué au fil du temps ? sur quelles épistémologies se fondent leurs pratiques ?
		  \end{itemize}

	\vspace{0.25cm}
	\end{alertblock}

\end{textblock*}



%%%%%%%%%%%%%%%%%%%%%%%%%%%%%%  Matériel et méthodes  %%%%%%%%%%%%%%%%%%%%%%%%%%%%%%%


\begin{textblock*}{\colwidth}(\leftmargin + \colwidth + \colsep, \blockOne)

	\begin{alertblock}{3 - Matériel et méthodes.}
	\RaggedRight
	\vspace{0.25cm}

	Ce mémoire s’appuie sur l'analyse d'une série d'entretiens réalisés dans le cadre du projet \textit{Oral Histories of Economics}, nous nous intéressons à la collection \cite{InterviewComputerizationEconomics}. Ces entretiens approfondis ont été réalisés avec David Hendry, Joshua Epstein, Robert Axtell, Theodore Turocy, Urs Fischbacher, Richard Pierse et Agnès Gramain, sept figures ayant contribué à la création de logiciels devenus centraux dans différents sous-champs de l’économie (économétrie, modélisation agent, théorie des jeux, expérimentation…). Ils portent sur leurs trajectoires personnelles, leurs expériences de création de logiciels, et leurs points de vue sur l'informatisation de l'économie 
	\vspace{0.5em}

	Nous nous appuierons également sur la littérature secondaire, portant sur les mêmes sujets. Elle contiendra des travaux de références documentant l’évolution et l’influence de l’informatique en économie ainsi que ses acteurs (entre autres, \cite{boumansComputerizationEconomicsThree2023}, \cite{cherrierWriteYourModel2023}, \cite{backhouseItsComputersStupid2017}).
	Ces travaux fourniront une base théorique pour comprendre comment la technologie croise la théorie et la pratique économique, permettant une contextualisation plus large des entretiens.
	
	\vspace{0.25cm}
	\end{alertblock}

\end{textblock*}



%%%%%%%%%%%%%%%%%%%%%%%%%%%%%%  Résultats principaux  %%%%%%%%%%%%%%%%%%%%%%%%%%%%%%%


\begin{textblock*}{\colwidth}(\leftmargin + 2\colwidth + 2\colsep, \blockOne)

	\begin{alertblock}{4 - Résultats principaux.}
	\RaggedRight
	\vspace{0.25cm}

	Avant d’aborder les résultats, il convient de se rappeler des limites inhérentes à la méthode de l'entretien. Ces derniers constituent une source située et construite, et il faut garder en tête lors de la lecture de ces résultats.
	
	Cependant, les trajectoires étudiées révèlent plusieurs traits communs : apprentissage souvent autodidacte de la programmation, débuts universitaires dans les sciences dites “exactes” (mathématiques, informatique), rapport pragmatique aux langages de programmation, importance du bricolage et de l’expérimentation. Tous les économistes-programmeurs enquêtés se distinguent par une posture artisanale face aux outils : ils les construisent quand ceux-ci n’existent pas, ou les adaptent à leurs besoins spécifiques.

	Des différences notables apparaissent néanmoins : générationnelles (des mainframe IBM aux PC d'aujourd'hui), institutionnelles (université, banque centrale, think tank) ou épistémiques (automatiser, simuler). Leurs logiciels ont souvent été conçus pour répondre à un besoin local, avant d’être diffusés plus largement. La reconnaissance académique reste inégale : certains logiciels deviennent des "citeware", cités dans chaque article qui les utilise, d’autres restent invisibles malgré leur impact.

	Ces différences générationnelles, les usages qui dominent aujourd'hui et la vitesse à laquelle l'informatique évolue et se recompose ouvrent des perspectives pour les prochains économiste-programmeur. Leurs pratiques seront plus tournées vers le développement de librairies spécialisées sur une tâche, s'appuyant sur des logiciels open source comme Python ou RStudio, permettant de laisser le travail informatique le plus complexe aux informaticiens, et aux économistes sur la conception de leurs outils.
	
	Enfin, une forte masculinisation du champ est manifeste : une seule femme parmi les sept personnes interrogées. Cela interroge les conditions d’accès à la programmation en économie et les biais structurels persistants.
	
	En somme, la programmation apparaît comme une pratique située, à la fois outil, méthode et engagement intellectuel, qui redéfinit les manières de faire de l’économie.

	\vspace{0.25cm}
	\end{alertblock}

\end{textblock*}




%%%%%%%%%%%%%%%%%%%%%%%%%%%%%%  Conclusion   %%%%%%%%%%%%%%%%%%%%%%%%%%%%%%%

\begin{textblock*}{2\colwidth + \colsep}(\leftmargin, \blockFour)

	\begin{alertblock}{5 - Conclusion.}
	\RaggedRight
	\vspace{0.25cm}

	L’étude des économistes-programmeurs éclaire une transformation profonde et encore peu analysée de la discipline économique : celle par laquelle le développement logiciel devient une modalité centrale de production des savoirs. Ces trajectoires révèlent une position professionnelle hybride, combinant formalisation, innovation logicielle et reconfiguration des pratiques scientifiques, engagée dans la conception d’outils qui transforment les objets, les méthodes et les normes de la recherche économique.

	Les projets logiciels naissent rarement de plans préétablis. Ils émergent par nécessité, opportunité ou bricolage intellectuel, dans les marges de la recherche ou de l’enseignement. Leur diffusion dépend ensuite de multiples facteurs : appropriation communautaire, reconnaissance académique, stabilité institutionnelle. Certains outils deviennent des standards (z-Tree, Ox, Sugarscape), d'autres restent invisibilisés malgré leur utilité.

	Ces trajectoires révèlent enfin une transformation de l’identité professionnelle : de simples utilisateurs d’outils, ces chercheurs deviennent producteurs de méthodes et porteurs d’une épistémologie propre. Leurs pratiques s’ancrent dans une logique d’explication générative, d'instrumentation ou d'épistémologie constructive, qui redéfinit les manières de faire de l’économie. La programmation devient ainsi une forme d’engagement scientifique à part entière.

	\vspace{0.25cm}
	\end{alertblock}

\end{textblock*}



%%%%%%%%%%%%%%%%%%%%%%%%%%%%%%  Bibliography  %%%%%%%%%%%%%%%%%%%%%%%%%%%%%%


\begin{textblock*}{\colwidth}(\leftmargin + 2\colwidth + 2\colsep, \blockFour)

\begin{parblock}{\textmd{\refname}}
	\vspace{-1.333333\baselineskip}  % Include this negative \vspace if starting a parblock with a list.
	\setlength{\bibhang}{0pt}
	\setlength{\bibsep}{0\baselineskip}
	\renewcommand*{\bibfont}{\sffamily\small\setlength{\baselineskip}{28.8pt}}
	%\renewcommand*{\bibfont}{\fontsize{23.5pt}{27.65pt}\sffamily}  % Adjust freely if necessary
	\renewcommand*{\bf}{\bfseries}
	\begin{refcontext}[sorting = nyt]
	\printbibliography[heading = none]
	\end{refcontext}

\end{parblock}

\end{textblock*}


\end{frame}


\end{document}