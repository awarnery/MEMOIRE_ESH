
Ce mémoire avait pour objectif d’analyser la place des logiciels dans la pratique économique, à travers l’étude de 7 entretiens réalisés avec des programmeurs-économistes. L’ambition était double : documenter un pan encore peu exploré de l’histoire récente de l’économie et proposer une analyse de l’informatisation de la discipline. Le recours à la méthode de l'histoire orale a été indispensable pour pemettre de saisir non seulement des trajectoires individuelles, mais aussi la manière dont les chercheurs eux-mêmes décrivent leurs rapports à l’informatique, à l'économie, aux langages de programmation et aux outils qu’ils ont contribué à développer. En effet, comme nous l'avons vu dans ce mémoire, ces pratiques transparaissent peu dans la littérature académique, du fait d'un manque de légitimité de celles-ci auprès des principaux journaux.

L’analyse de ces témoignages met en évidence plusieurs résultats. Tout d’abord, l’informatisation de l’économie est portée par des chercheurs souvent issus des sciences dites « dures » (mathématiques, informatique, ingénierie, physique) et formés à la programmation qui s'intéressent à l'économie dans un second temps et y importent leurs compétences techniques. Cette hybridité disciplinaire a joué un rôle déterminant dans la capacité à créer des logiciels robustes et complexes à une époque où l’informatique restait instable et rudimentaire. Ensuite, le développement de ces outils précurseurs s’est accompagné d’un travail considérable d’ingénierie logicielle, bien au-delà de la seule dimension économique. Enfin, l’évolution récente des environnements informatiques a profondément transformé la pratique : l’essor des langages de haut niveau et des écosystèmes open source (R, Python) permet désormais aux économistes d’utiliser et de produire des outils sans avoir à réinventer les fondations logicielles. Ces innovations technologiques permettent aujourd'hui l'avènement d'économistes-programmeurs, se définissant d’abord comme économistes, et seulement ensuite comme programmeurs.

Au-delà de ces constats empiriques, cette enquête conduit à une réflexion plus générale sur la dynamique de l’informatisation de l’économie. Trois leçons principales se dégagent.

Premièrement, l’économie s’inspire abondamment des sciences dures pour se développer, mais cette appropriation ne concerne pas uniquement des compétences techniques (programmation, simulation), elle s'étend aussi aux cadres conceptuels. Comme l’ont analysé Mirowski dans \textit{Machine Dreams : Economics Becomes a Cyborg Science}\cite{mirowskiMachineDreamsEconomics2001} et \textit{More Heat than Light : Economics as Social Physics, Physics as Nature's Economics}\cite{mirowskiMoreHeatLight1989} et McCloskey dans \textit{the rethoric of Economics}\cite{mccloskeyRhetoricEconomics1983}, l’économie a une longue tradition d’emprunt aux mathématiques, à la physique ou à la biologie, pour construire des métaphores et des analogies qui structurent ses modèles. 

Deuxièmement, cette ouverture permanente se combine, paradoxalement, à une inertie marquée dans certaines pratiques. Comme en témoigne Agnès Gramain dans son entretien, certains enseignants, dont elle fait partie, ont du mal à se défaire des logiciels propriétaires comme Stata ou SAS. Ces logiciels continuent d’occuper une place importante, non en raison de leurs performances, mais parce que les enseignants y sont habitués et continuent de les transmettre lors de leurs enseignements. Cette inertie institutionnelle limite la diffusion des outils open source comme R et Python.

Troisièmement, les programmeurs-économistes, dans leurs entretiens, confirment une forme de conservatisme générationnel, résumé par le principe de Planck : dans l’économie comme dans d’autres disciplines, les véritables changements ne s’imposent qu’avec le renouvellement des cohortes de chercheurs, “un décès à la fois”. L’économie apparaît donc à la fois comme une discipline ouverte, nourrie par les sciences "dures" et leurs métaphores, et comme une discipline résistante au changement, où l’innovation technologique et logicielle se heurte à des routines et à une adoption lente de la part des chercheurs les plus anciens.

Ce travail présente cependant plusieurs limites. Le corpus d’entretiens reste restreint et centré sur une génération de programmeurs ayant soutenu leurs thèses et produit leurs logiciels entre les années 1970 et 2000. Les trajectoires plus récentes, par exemple celles de développeurs de packages contemporains comme Clément de Chaisemartin ou des équipes de microsimulation de l’IPP, mériteraient d’être étudiées pour mieux comprendre comment les jeunes générations mobilisent les outils informatiques modernes de l'économiste. Par ailleurs, les entretiens, par nature, donnent accès à des récits situés et subjectifs. Ces limites de la méthode de l'histoire orale sont donc à garder à l'esprit.

L’étude de la microsimulation, technique intrinsèquement liée à l’informatique et désormais centrale dans l’expertise économique des politiques publiques, permettrait par exemple d’élargir la réflexion sur le rôle de l’informatique dans la construction de la figure de l'expert économique. De même, l’arrivée des grands modèles de langage (LLM) et des assistants de programmation automatisés invite à s’interroger sur les nouvelles couches d’abstraction qui transforment l’accès à la programmation et, potentiellement, le rapport des économistes à leurs outils.

En définitive, l’informatisation de l’économie n’apparaît pas comme un simple ajout d’outils à une discipline déjà constituée, mais comme une transformation profonde des manières de travailler, de modéliser et d’enseigner. Elle révèle une tension permanente : celle d’une discipline toujours à la frontière de l’innovation, mais en même temps freinée par ses routines et son conservatisme.