

I. Présentation des logiciels développés par les économistes-programmeurs
L’analyse du corpus met en lumière une série de logiciels qui, loin d’être de simples instruments auxiliaires, constituent de véritables artefacts épistémiques. Ils matérialisent une certaine conception de l’économie et en redéfinissent les pratiques, qu’il s’agisse de l’enseignement de l’économétrie, de la simulation macroéconomique, de l’expérimentation en laboratoire, de la théorie des jeux computationnelle ou encore de la modélisation multi-agents. La présente section revient, pour chacun des économistes étudiés, sur l’histoire, les objectifs et l’impact des logiciels qu’ils ont conçus ou utilisés.

1. David Hendry et l’écosystème PcGive / OxMetrics
L’expérience logicielle de David Hendry illustre le passage, dans les années 1980, d’une pratique de l’économétrie confinée aux mainframes à un univers micro-informatique centré sur l’accessibilité et la robustesse. Formé à la programmation en Fortran dès son doctorat à la LSE, Hendry développe à Oxford, dans les années 1980, la bibliothèque Autoreg, composée de sous-routines permettant l’estimation d’un large éventail de modèles, la réalisation de simulations Monte Carlo et la génération de données artificielles. L’arrivée du PC IBM en 1981 constitue un tournant : convaincu que la visualisation et l’interactivité deviendront des dimensions essentielles de la pratique économétrique, Hendry entreprend la transposition de son environnement de travail sur micro-ordinateur.
C’est dans ce contexte qu’apparaît PcGive, premier logiciel de la série, pensé comme un outil pédagogique autant que scientifique. Contrairement à la logique du « batch processing » des mainframes, PcGive introduit une interface guidée par menus, permettant aux étudiants comme aux chercheurs d’explorer différentes procédures économétriques sans écrire eux-mêmes les routines. Le logiciel vise explicitement à transformer l’enseignement, en donnant la possibilité de montrer « en direct » l’effet d’un modèle mal spécifié ou d’un changement structurel, grâce à des simulations répétées.
L’évolution technique du projet est marquée par la rencontre avec Jurgen Doornik, qui assure la réécriture du système en C et introduit le langage Ox. Celui-ci offre une souplesse inédite : le code généré automatiquement par PcGive peut être sauvegardé et réutilisé comme script indépendant, garantissant à la fois la reproductibilité et l’extensibilité des analyses. Autour de Hendry et Doornik se constitue une véritable équipe de développeurs (Hans-Martin Krolzig pour PcGets, Siem Jan Koopman pour STAMP, Sebastian Laurent pour G@rch), donnant naissance à une suite logicielle complète, connue sous le nom d’OxMetrics.
L’impact de cette suite a été considérable. OxMetrics s’est imposé comme un standard international, diffusé dans de nombreux programmes doctoraux et largement adopté dans la recherche appliquée. Si certains ont pu critiquer une forme de « boîte noire » qui réduirait le rôle de l’économètre, Hendry défend au contraire une conception où la robustesse et la pédagogie priment sur l’opacité des routines artisanales. PcGive et ses prolongements constituent ainsi l’un des cas les plus emblématiques d’un logiciel qui a redéfini l’économétrie comme pratique académique et comme discipline enseignée.

2. Richard Pierse et WinSolve
Le parcours de Richard Pierse se situe dans la continuité de cette tradition oxfordienne de logiciels économétriques, mais en la réorientant vers la simulation macroéconomique. Après avoir travaillé sur le logiciel NIModel au National Institute (NIESR), utilisé dans plusieurs institutions britanniques, Pierse développe dans les années 1990 WinSolve, un solveur généraliste pour les modèles macroéconomiques de grande taille.
L’objectif de WinSolve est double : d’une part, fournir aux chercheurs un outil flexible permettant de manipuler et de simuler des modèles comportant plusieurs centaines d’équations ; d’autre part, offrir aux institutions (Banque d’Angleterre, Trésor, laboratoires universitaires) un logiciel standardisé qui évite la dépendance à des codes locaux difficiles à maintenir. Financé notamment par des projets de l’ESRC, WinSolve est pensé comme une infrastructure collective au service de la communauté.
Sur le plan technique, WinSolve introduit des algorithmes efficaces de résolution numérique, y compris pour des modèles non linéaires avec anticipations rationnelles. Il permet l’exploration interactive de scénarios de politique économique, et s’impose ainsi comme un outil incontournable dans l’écosystème britannique de modélisation. Comparé à d’autres environnements (TROLL, puis Dynare), WinSolve conserve une logique académique et ouverte, inscrite dans la mission de service public de la recherche économique.

3. Theodore Turocy et Gambit
Le logiciel Gambit constitue un cas emblématique de la rencontre entre la théorie des jeux et la programmation. Lancé au début des années 1990 par Richard McKelvey et Andrew McLennan, le projet bénéficie très tôt de la contribution de Theodore Turocy, alors étudiant à Caltech. D’abord simple collaborateur, Turocy deviendra le principal architecte et mainteneur du logiciel, assurant sa survie bien au-delà du financement initial de la NSF.
Gambit vise à offrir une bibliothèque computationnelle ouverte pour représenter et analyser des jeux, sous forme normale ou extensive, et calculer différents types d’équilibres (Nash, epsilon-équilibres, etc.). Sa conception repose sur une architecture modulaire et sur la volonté de constituer une référence publique, à contre-courant d’une littérature où les calculs restaient souvent implicites ou inaccessibles.
Malgré son importance scientifique et pédagogique, Gambit a souffert d’un manque chronique de financements après 2000. Maintenu sur la base du volontariat par Turocy, il a trouvé un nouveau souffle dans le cadre de l’Alan Turing Institute, où ses applications en intelligence artificielle (agents autonomes, enchères en ligne, défense informatique) ont permis de renouveler sa légitimité. Gambit illustre ainsi la difficulté à inscrire durablement un logiciel académique dans l’écosystème économique, tout en soulignant son potentiel interdisciplinaire.

4. Urs Fischbacher et z-Tree
Développé à partir de 1995 à l’Université de Zurich, z-Tree (Zurich Toolbox for Readymade Economic Experiments) répond à un besoin très concret : disposer d’un logiciel fiable pour organiser des expériences économiques avec des participants humains. Recruté comme programmeur par Ernst Fehr, Urs Fischbacher en assure la conception et l’implémentation, avant de s’impliquer pleinement dans la recherche en économie expérimentale.
L’objectif du projet est clair : standardiser la conduite des expériences et rendre possible la comparaison entre laboratoires. z-Tree offre une interface simple pour programmer des jeux expérimentaux, tout en s’appuyant sur une architecture client-serveur robuste. Le choix d’un environnement Windows se révèle stratégique : il garantit une compatibilité maximale et facilite la diffusion du logiciel à l’échelle mondiale.
Rapidement, z-Tree s’impose comme le standard de facto de l’économie expérimentale. Sa souplesse permet de reproduire des protocoles complexes, impossibles à gérer manuellement, et contribue à l’essor du champ. Le logiciel a profondément transformé la pratique expérimentale : il réduit les coûts, accroît la reproductibilité, et constitue un langage commun pour une communauté en expansion rapide. Sa dépendance à un environnement propriétaire (Windows) et à des scripts peu transparents a néanmoins suscité des débats, révélant la tension entre efficacité technique et exigences de transparence scientifique.


5. Joshua Epstein et Robert Axtell et Sugarscape
Parmi les logiciels étudiés, Sugarscape occupe une place singulière : il marque l’entrée de la modélisation multi-agents dans les sciences sociales. Né en 1992 dans le cadre du 2050 Project financé par la Fondation MacArthur, il est conçu par Joshua Epstein et Robert Axtell au Brookings Institution, en dialogue avec le Santa Fe Institute.
L’anecdote de sa genèse est devenue légendaire : le modèle est d’abord esquissé sur une serviette à la cafétéria de Brookings, puis rapidement codé par Axtell. Le premier prototype consiste en un paysage peuplé d’agents récoltant du « sucre ». Dès ce stade embryonnaire, le logiciel incarne une ambition novatrice : créer une « artificial social life » permettant d’explorer l’émergence de phénomènes macroscopiques (distribution des richesses, ségrégation spatiale, propagation d’épidémies) à partir d’interactions locales simples.
L’innovation majeure de Sugarscape réside dans l’épistémologie générative : pour Epstein, « si vous ne pouvez pas le faire émerger, vous ne l’avez pas expliqué ». Le logiciel devient ainsi le manifeste d’une nouvelle approche de l’explication en sciences sociales, reposant sur la simulation d’agents cognitivement plausibles.
La publication de Growing Artificial Societies (1996) consacre Sugarscape comme un jalon fondateur. Ses applications se multiplient, de l’économie à l’épidémiologie, en passant par l’archéologie et les sciences politiques. Souvent critiqué pour la simplicité de ses hypothèses, le modèle a néanmoins démontré une puissance heuristique considérable et ouvert la voie à l’essor de la modélisation agent-based dans les sciences sociales.


6. Agnès Gramain : une figure différente de l’« économiste programmeur »
À la différence des autres cas, Agnès Gramain n’a pas créé de logiciel diffusé à grande échelle. Son expérience témoigne d’une autre modalité de l’informatisation de l’économie : l’appropriation et l’adaptation d’outils existants.
Ses recherches mobilisent intensivement des logiciels statistiques comme SAS et GAUSS, auxquels elle adjoint des routines spécifiques, écrites pour répondre à des besoins ponctuels. Elle développe ainsi une pratique de « bricolage » qui, si elle ne conduit pas à la création d’un outil collectif, n’en révèle pas moins l’importance quotidienne de la programmation dans le travail de recherche. Parallèlement, l’usage avancé de TeX/LaTeX témoigne d’une maîtrise technique au service de la diffusion scientifique.
Ce profil met en lumière une autre figure de l’« économiste programmeur » : non pas l’inventeur d’un standard disciplinaire, mais l’utilisateur-créateur qui étend les capacités de logiciels existants pour répondre à des besoins spécifiques. La trajectoire de Gramain rappelle que la programmation peut être constitutive du travail scientifique même en l’absence de logiciel emblématique, et invite à penser l’« informatisation » de l’économie dans une acception élargie.


