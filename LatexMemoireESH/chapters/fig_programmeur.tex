
Avant de commencer  l'analyse de nos différents personnages, il est important d'avoir en tête quelques éléments de leurs biographies.


David Hendry

Économètre britannique, formé en économie et en économétrie, il soutient son doctorat en 1970. Hendry influence profondément l’économétrie empirique en développant des outils logiciels (PcGive, OxMetrics) pour automatiser et fiabiliser les tests économétriques. Professeur à Oxford, il joue un rôle majeur dans la structuration de l’économétrie comme domaine de l'économie au Royaume-Uni.


Joshua Epstein

Chercheur américain, formé en mathématiques et en sciences politiques, il soutient son doctorat en 1981. Il est un pionnier de la modélisation agent-based en sciences sociales, notamment avec le modèle Sugarscape co-développé avec Robert Axtell. Il mène sa carrière entre think tanks, épidémiologie, économie et recherche interdisciplinaire, plaidant pour une épistémologie générative.


Richard Pierse

Économiste britannique, après un bachelor à Oxford en philosophie, politique et économie, il est diplômé en 1979 d'un master en économétrie de la London School of Economics (LSE). Richard Pierse commence un doctorat en économétrie qu'il n'achève pas, faute de financements. Il est l’auteur des logiciels NIModels et WinSolve, des outils de simulation macroéconomique appliqués à la politique monétaire, très utilisés dans les banques centrales.


Urs Fischbacher

Chercheur suisse, il soutient un doctorat en mathématiques théoriques en 1985 puis travaille comme ingénieur logiciel. Il devient ensuite une figure clé de l’économie expérimentale grâce à la création de z-Tree, un logiciel de référence pour la conduite d’expériences en laboratoire, aujourd’hui utilisé dans le monde entier.


Robert Axtell

Chercheur américain formé à l’ingénierie, il se spécialise ensuite sur les thématiques aux croisements de l’économie, de l’informatique et des politiques publiques. Il soutient son doctorat en 1992. Il co-développe avec Joshua Epstein le modèle Sugarscape, pierre fondatrice de la modélisation multi-agents en sciences sociales.


Agnès Gramain

Formée à l’ENSAE, elle est une chercheuse française spécialiste de microéconomie appliquée. Elle obtient son doctorat en 1998 en économie de la santé, domaine qu'elle continue d'explorer dans ses recherches par la suite.


Theodore Turocy

Chercheur américain, diplômé de Caltech avec une majeure en informatique appliquée et une mineure en sciences sociales, il soutient un doctorat en 2001 sur les enchères et les méthodes computationnelles pour déterminer les équilibres en situation complexe. Il se spécialise dans la théorie des jeux computationnelle. Développeur principal du logiciel Gambit, il combine recherche académique, enseignement et projets d’intelligence artificielle appliquée à la décision stratégique.


%----------------------------------------------------------------------------------------
%	SECTION 1
%----------------------------------------------------------------------------------------
\section{Les traits communs partagés par les programmeurs-économistes}



\subsection{Des formations académiques axées sur les sciences dites "exactes"}

Le premier trait commun frappant, à la lecture de ces biographies, est la domination des formations académiques fortement mathématisée.
5 des 7 programmeurs-économistes dont nous analysons ici les trajectoires, ont commencé leurs parcours académiques dans des disciplines dites "exactes", souvent exigeantes sur le plan mathématique, l’implication dans l’économie ou les sciences sociales, venant dans un mouvement de bifurcation intellectuelle guidé par la volonté de comprendre les phénomènes sociaux à l’aide d’outils formels. 
Cette hybridité se reflète dans leurs objets d’étude (théorie des jeux, économétrie, dynamiques sociales, expérimentations) et dans leur manière d’aborder les problèmes. 

Turocy incarne typiquement ce profil hybride : diplômé de Caltech, il y suit un double cursus en informatique appliquée et en sciences sociales, avant de s’orienter vers la théorie des jeux computationnelle. Dès ses premières années, il développe des outils logiciels et participe à des projets de recherche expérimentale. Il est également intéressant de noter que McKelvey et McLennan, les deux créateurs du logiciel Gambit, sont mathématiciens de formations, avant qu'ils ne s'intéressent aux applications de la théorie des jeux en économie.

Epstein commence par la musique, puis se passionne pour les mathématiques pures, qu’il cherche ensuite à appliquer à l’étude des systèmes sociaux. Il choisit de faire un doctorat en sciences politiques au MIT, avec une forte composante en économie. C’est dans ce cadre qu’il développe ses premiers modèles mathématiques, avant de passer à la simulation agent-based, notamment avec le projet Sugarscape. Axtell, avec qui il développe ce logiciel, a une formation d'ingénieur.

Hendry suit initialement une formation en économie, avant d’être orienté vers l’économétrie sous l’influence décisive de Denis Sargan, son mentor à la London School of Economics. Il devient ensuite une figure centrale de l’économétrie au Royaume-Uni. Nous n'avons pas pu trouver d'informations sur la formation de Jurgen Doornik, qui accomplie la majeure partie du travail technique sur les logiciels d'Hendry, mais sa thèse est soutenu en "Econometric computing".

Fischbacher étudie les mathématiques théoriques à l’université, soutien un doctorat dans cette discipline, puis travaille brièvement comme ingénieur logiciel. C’est par hasard, du fait d’une opportunité professionnelle, qu’il intègre un laboratoire d’économie expérimentale. Il y découvre un champ en construction, où ses compétences en programmation se révèlent précieuses, ce qui l’amène à concevoir le logiciel z-Tree et à s’ancrer durablement dans l’économie expérimentale. Il est ensuite rejoint dans son travail sur z-Tree par Stefan Schmid, formé en ingénierie logicielle à l’ETH Zurich, une école d'ingénieur.

Pierse et Gramain font un peu figure d'exception. Le premier a une formation en Philosophie, Politique et Économie (PPE), une filière généraliste bien connue au Royaume-Uni. Il se passionne pour la programmation sur son temps libre et c'est lors de ses emplois d’été au National Institute of Economic and Social Research (NIESR) qu'il commence à pratiquer l'informatique appliquée à l'économie. Gramain commence ses études par une classe préparatoire B/L, très pluridisciplinaire (Philosophie, Histoire, Economie, Mathématiques). Elle intègre l’ENSAE, une école d'ingénieur, où elle est formée à la fois à l’économie, aux statistiques et aux mathématiques appliquées. C’est dans le cadre de ses premiers travaux que la programmation devient un outil indispensable.

D'ailleurs certains doutent même de leur légitimité en tant qu'économistes :

\begin{quote}
\begin{center}
\textit{"So I did my habilitation in 2006. And actually I did that because I have no other formal degree in economics. So I thought I need to kind of prove that I also have something in economics."}
\end{center}
\end{quote} \hfill Entretien de Urs Fischbacher\cite{fischbacherInterviewUrsFischbacher2024}, page 5

Urs Fischbacher a donc fait son habilitation avant tout pour acquérir un diplôme d'économie et asseoir sa légitimité dans cette discipline.

Joshua Epstein à propos de lui-même et d'Axtell, quand on lui demande ce qu'il pense des travaux de Kanheman et Tversky :

\begin{quote}
\begin{center}
\textit{"I was, kind of illiterate. I hadn’t really studied Kahneman or anybody. Rob’s
background had been chemical engineering and mine had been sort of social science, mathematical
modeling ..."}
\end{center}
\end{quote} \hfill Entretien de Epstein\cite{epsteinGrowingArtificialSocieties1996}, page 8

Et Turocy sur son positionnement vis à vis de l'économie et des sciences de l'ingénieur :

\begin{quote}
\begin{center}
\textit{"I’ve pretty much just been very open now about the fact that I’m an engineer, who happens to be interested in social science problems. [...] I definitely have made my peace, I have decided to identify [...] as an engineer principally"}
\break
\textit{"I am an engineer at heart"}
\end{center}
\end{quote} \hfill Entretien de Turocy\cite{turocyInterviewTheodoreTurocy2024}, page 8 et page 31


Malgré des parcours variés, tous partagent un socle scientifique solide, principalement en mathématiques, qui les prépare à mobiliser des outils informatiques. Fischbacher poursuit un doctorat en mathématiques ; Epstein soutient en sciences politiques, Hendry, Turocy et Gramain en économie avec une forte composante d'informatique. Pierse, quant à lui, ne termine pas son doctorat d'économétrie, faute de financement, bien qu’il mène une carrière active dans le monde institutionnel et à la Banque d’Angleterre en tant que chercheur.

Ce qui les réunit, c’est que leurs premiers travaux de recherche mobilisent immédiatement l’informatique, que ce soit pour simuler, expérimenter ou automatiser. L’hybridité disciplinaire n’est pas un détour mais un levier essentiel de leurs trajectoires : c’est précisément parce qu’ils viennent d’ailleurs qu’ils ont apporté à l’économie des outils et des méthodes computationnelles innovantes.

Il est également intéressant de constater le même phénomène dans l’histoire de logiciels extérieurs à notre corpus.
Ainsi, le développement de Dynare, logiciel d’estimation de modèles d’équilibre général dynamique stochastique, étudié par Cherrier et al. \cite{cherrierWriteYourModel2023}, est aussi marqué par ce genre de profil :

\begin{quote}
\begin{center}
\textit{"While Malgrange and Pujol worked on the theoretical structure of the model, Laffargue focused on the methods for solving and simulating the model."}
\end{center}
\end{quote} \hfill Histoire de Dynare\cite{cherrierWriteYourModel2023}, page 11

Il souligne lui-même, comme plusieurs des programmeurs dont nous étudions les entretiens, que sa formation d'ingénieur civil de l’aéronautique le préparait particulièrement à ce type de tâche. Il se dit "ingénieur" et non pas "économiste", c'est son identité professionnelle :

\begin{quote}
\begin{center}
\textit{"For engineers and physicists, differential equations with initial and terminal conditions are a common problem, which had been known and dealt with since a long time ... And, myself, I was an engineer.” Laffargue recalls."}
\end{center}
\end{quote} \hfill Histoire de Dynare\cite{cherrierWriteYourModel2023}, note en bas de page 12

De façon similaire, les créateurs du logiciel de microsimulation socio-fiscale OpenFisca ont commencé leurs parcours académiques dans les sciences dites "dures" :

\begin{quote}
\begin{center}
\textit{"Ce sont deux chercheurs en économie de France Stratégie – une institution rattachée au Premier ministre en charge de déterminer les grandes orientations stratégiques du pays qui sont à l’origine d’OpenFisca. Il s’agit de Mahdi B., docteur en physique puis chercheur en économie à la Banque de France et à France Stratégie avant de rejoindre l’IPP, et de Nicolas A., polytechnicien et économiste [...]"}
\end{center}
\end{quote} \hfill Histoire d'OpenFisca\cite{shulzLogicielLibrePour2019}, page 10

Ces cas confirment que l’introduction de nouvelles pratiques, de nouveaux outils informatiques, en économie s’appuie fréquemment sur des chercheurs issus de disciplines voisines, capables de transposer leurs compétences en mathématiques, physique ou ingénierie dans le champ économique.





\subsection{Un apprentissage autodidacte de l'informatique}

Un autre trait partagé par la majorité des programmeurs-économistes interrogés est leur apprentissage autodidacte de l’informatique et des langages de programmation. Tous ont acquis leurs compétences par nécessité, par expérimentation, ou par curiosité, dans des environnements qui rendaient cet apprentissage possible. Theodore Turocy illustre parfaitement cette trajectoire : il commence à coder dès l’enfance, sur un TI-99/4A offert par grand-père. Il explore de lui-même différents langages, jouant avec les outils disponibles à la maison ou à l’école. Cette familiarité précoce avec la machine s’inscrit dans un contexte nord-américain où l’accès aux micro-ordinateurs domestiques, bien que coûteux, est en pleine démocratisation, grâce aux politiques publiques de subventions. Il semble aussi que le niveau socio-économique soit un fort prédicteur de l'adoption de l'ordinateur dans les foyers (Bureau of Labor Statistics 1999\cite{ComputerOwnershipSharply}, Schmitt 2002\cite{schmittGivePCsChange2002}). 

David Hendry, de son côté, découvre l’informatique à l’université, à l’époque des premiers ordinateurs centralisés, les "mainframe". Ces ordinateurs occupent une pièce entière, et le code doit être inscrit sur des cartes cartons à trous pour qu'il soit lu par l'ordinateur. Il relate une expérience frustrante lorsqu'il essaye d'apprendre Atlas Autocode, un langage spécifiquement prévu pour l'ordinateur Atlas du Center of London University, qui lui semble très déconcertant. Il se tourne ensuite vers Fortran, le langage le plus utilisé à l'époque, car pouvant être utilisé sur les ordinateurs mainframe d'IBM, les plus performants de cette époque. Il reçoit alors quelques cours dans le cadre de son doctorat, dirigés par Carol Hewlett. Ce sont les échanges entre pairs et les essais erreurs qui forment le cœur de sa progression. 

Joshua Epstein, bien que formé en mathématiques et en sciences sociales, n’a pas reçu d’enseignement en programmation. C’est dans le cadre du projet Sugarscape, au début des années 1990, qu’il découvre de manière pragmatique la modélisation informatique, aux côtés de Robert Axtell. À l’époque, aucun outil dédié à la modélisation agent n’existait : tous deux doivent tout construire eux-mêmes, à partir de langages de bas niveau. \footnote{Pour comprendre la différence entre langage de bas niveau et langage de haut niveau, on peut utiliser la métaphore de la construction d’une maison. Programmer en bas niveau revient à poser chaque brique soi-même : on contrôle précisément chaque étape, on comprend comment tout fonctionne, mais cela demande du temps et beaucoup d’efforts. À l’inverse, programmer en haut niveau, c’est comme assembler une maison à partir d’éléments préfabriqués : on avance plus vite, car beaucoup d'éléments sont fabriqués par d'autres personnes, mais on perd en transparence et en maîtrise fine de la structure, on ne sait pas en détail comment sont fabriqués ces éléments préfabriqués.} 

Urs Fischbacher et Richard Pierse montrent également des trajectoires d’apprentissage autonome et pragmatique, acquérant leurs compétences informatiques principalement à travers des ordinateurs mis à disposition dans leurs lycées. 

Enfin, Agnès Gramain raconte avoir commencé à programmer en SAS à son entrée à l’ENSAE. Ce sont les habitudes et les manuels (les "pieuvres" rédigée par un administrateur) de programmation développé à l’INSEE, qui sont mobilisés dans cette école formant les futurs cadres de l’institut. C'est donc la seule programmeuse-économiste qui s'inscrit dans une formation par et pour une institution. 

Dans tous les cas, cette autoformation s’est déroulée dans des environnements techniquement bien équipés : universités disposant de laboratoires informatiques, accès précoce à des machines coûteuses, parfois dès le lycée ou à domicile. Ces conditions matérielles, souvent invisibilisées, ont facilité l’exploration et l’appropriation des outils numériques, à une époque où la documentation était rare et les interfaces peu conviviales. L’apprentissage de la programmation n’est donc pas seulement une affaire de volonté individuelle : il est aussi rendu possible par des conditions d’accès matérielles et institutionnelles favorables, dans des contextes où la curiosité scientifique pouvait s’exprimer librement. Le système PLATO en est un bon exemple (Lee p.10 2004\cite{leeHistoryComputingEducation2004}).


\subsection{Une certaine souplesse dans les langages utilisés}

Un autre trait commun à ces programmeurs-économistes est leur souplesse dans l’usage des langages de programmation. Aucun d’entre eux ne témoigne d'une fidélité durable à un langage unique. Tous adoptent une démarche pragmatique, choisissant leurs outils en fonction des contraintes techniques, des projets en cours, ou des évolutions technologiques. 

David Hendry commence à programmer en Fortran, langage standard des années 1970-80 dans les milieux économétriques. Par la suite, il collabore étroitement avec Jurgen Doornik, qui réécrit une grande partie de leur outil en C, avant de lui déléguer le développement du langage Ox, spécifiquement conçu pour l’économétrie appliquée et pour leurs logiciels. Cette évolution illustre une adaptation constante aux enjeux de performance et de reproductibilité. 

Theodore Turocy commence à coder très jeune sur un TI-99/4A familial, l'un des premiers ordinateurs familiaux, sur lequel il apprend le BASIC, puis a utilisé d'autres langages comme le C, Maple, Python et d’autres outils selon les besoins. Il contribue également au développement de Gambit, un environnement dédié à la théorie des jeux computationnelle. Sa trajectoire montre une grande aisance à naviguer entre langages, dans une logique d’expérimentation et d’efficacité. 

Joshua Epstein, bien que formé initialement à la modélisation mathématique “papier-crayon”, s’est rapidement tourné vers des langages comme C++ pour développer Sugarscape au début des années 1990, à une époque où aucun environnement de modélisation agent n’existait encore. Par la suite, il adopte des plateformes comme Repast ou NetLogo, dans un souci de portabilité et de diffusion. Urs Fischbacher conçoit pour sa part le logiciel z-Tree en C++, qu’il continue d’utiliser pour des raisons de performance, de portabilité et de stabilité. Il utilise aussi beaucoup Pascal. 

Fischbacher souligne l’importance de pouvoir tout coder lui-même, y compris l’interface, afin d’assurer une compatibilité et une fiabilité maximales dans les laboratoires d’économie expérimentale. 

Richard Pierse apprend d’abord à programmer en Fortran dans le cadre de son doctorat à Cambridge. Il adopte ensuite GAUSS, lors de son recrutement à Cambridge. Un langage très utilisé dans les années 1980-1990 pour l’économétrie, notamment pour sa souplesse dans la résolution de systèmes non linéaires et l’analyse de séries temporelles. Par la suite, il est contraint d'apprendre à utiliser C++ pour pouvoir construire son logiciel, WinSolve. 

Agnès Gramain fait figure d’exception : formée à SAS pendant sa scolarité à l’ENSAE, elle développe une forte affinité avec ce langage et affirme que ce dernier a structuré la manière dont elle pense, mais nous développerons cet aspect dans le paragraphe suivant. Au-delà des langages eux-mêmes, ces trajectoires illustrent une approche pragmatique et artisanale du développement logiciel : ce qui compte, ce n’est pas le langage en soi, mais la logique sous-jacente, ce qu’il permet de faire. Cependant, l’expérience d’Agnès Gramain semble montrer que les habitudes et les philosophies peuvent être chamboulées par un changement de langage.

\subsection{Une pratique très masculine}

Un autre trait frappant des trajectoires étudiées est le caractère très masculin du milieu de la programmation en économie. Parmi les 7 programmeurs-économistes interviewés, il n y qu'une seule femme : Agnès Gramain. Celle-ci évoque elle-même l’environnement masculin dans lequel elle a été formée : tant à l’ENSAE qu’au sein des laboratoires où elle développe ses compétences en simulation, les femmes sont rares. Elle ne rapporte pas d’obstacles explicites, mais son témoignage suggère que la programmation reste, dans son parcours, un domaine majoritairement investi par des hommes, où il faut trouver sa place sans beaucoup de modèles féminins. 

L’absence d’autres femmes parmi les personnes interrogées est peut-être aussi dû à la petite taille de notre échantillon, mais elle reflète une réalité plus large : l’informatique et l’économie sont deux disciplines historiquement très masculines, et leur point d’intersection, celui de la programmation en économie, l’est a fortiori. La littérature établit déjà clairement l'écart entre le nombre de femmes dans la population et le nombre de femmes étudiant l'informatique (Beyer, 2014\cite{beyerWhyAreWomen2014}) ou faisant de la recherche en informatique (Falkner et al., 2015\cite{falknerGenderGapAcademia2015}). Le même constat peut être fait en économie (Kahn, 1993\cite{kahnGenderDifferencesAcademic1993} et Bateman, 2023\cite{batemanGenderGapUK2023}). Cette sous-représentation manifeste invite à interroger les rapports de genre dans l’accès aux compétences informatiques et aux carrières scientifiques situées à l’interface entre économie et informatique. Plusieurs hypothèses peuvent être avancées : cette situation est-elle le résultat de biais historiques de recrutement, de barrières d’entrée structurelles à l’apprentissage de l’informatique pour les femmes, ou d’une construction genrée des compétences techniques, socialement valorisées chez les hommes et moins encouragées chez les femmes ? 

Le cas isolé d’Agnès Gramain ne permet pas à lui seul de tirer des conclusions définitives, mais il souligne la nécessité d’une réflexion plus large sur les inégalités de genre dans les pratiques informatiques en économie. Le fait que les programmeurs-économistes soient presque exclusivement des hommes dans cette enquête dit quelque chose des dynamiques d’exclusion ou d’auto-sélection à l’œuvre, encore aujourd’hui, dans l’économie computationnelle.




\subsection{Des moments de frustration}

Bien que tous les programmeurs-économistes interrogés valorisent la puissance de l’informatique dans leurs recherches, leurs trajectoires sont aussi marquées par des moments de frustration, de blocage ou de découragement face aux défis techniques, matériels ou institutionnels liés à l’usage de la programmation. 

David Hendry évoque les difficultés de ses débuts, au moment de la transition entre les ordinateurs centraux (mainframes) et les premiers ordinateurs personnels. Il relate les complications liées au langage Fortran, aux limites matérielles des machines de l’époque, ainsi qu’à l’absence d’environnement de travail ergonomique. La lenteur des calculs ou les erreurs de compilation complexes à diagnostiquer ont constitué des obstacles importants pour ses premiers travaux économétriques. 

Joshua Epstein, de son côté, insiste sur la rudesse des débuts de la modélisation agent-based, à une époque où aucun environnement logiciel dédié n’existait encore. Lors de la conception de Sugarscape, il travaille avec Rob Axtell sans bibliothèque\footnote{ensemble de fonctions ou routines pré-écrites, réutilisables pour faciliter le développement d'applications informatiques.} et avec très peu de documentation. Il décrit cette période comme exaltante mais techniquement aride, marquée par une forme de bricolage permanent et de navigation à vue dans un champ scientifique encore inexistant. 

Urs Fischbacher, qui développe z-Tree presque seul, souligne les difficultés techniques liées au développement et au maintien du logiciel : il doit tout concevoir, y compris l’interface graphique permettant l’interaction en temps réel entre les joueurs. 

Richard Pierse mentionne également des problèmes techniques concrets dans son travail à la Banque d’Angleterre. Lors de son second passage au sein de cette institution, il réalise notamment que trois versions de son programme NIModels ont évolué dans des directions les rendant désormais incompatibles entre elles. Il témoigne aussi de difficultés lors de l’utilisation de bases de données économiques volumineuses, ou dans la mise en œuvre de simulations complexes nécessitant une puissance de calcul importante. Ces défis sont amplifiés par la nécessité de produire des résultats exploitables dans des délais courts, au service de la politique monétaire. 

Agnès Gramain, elle, vit très mal le fait d'être contrainte d’utiliser GAUSS pendant sa thèse, pour des questions de performances, alors qu'elle avait appris à maitriser SAS pendant sa formation universitaire. Elle essaye maintenant d'apprendre à utiliser R, un logiciel gratuit, libre et donc plus accessible à ses étudiants. Mais témoigne retourner vers SAS lorsqu'elle doit faire de la recherche : 
\begin{quote}
\begin{center}
\textit{"Mais c’est comme une langue maternelle. Je pense qu’une fois que tu as appris une langue, ça structure ta manière de penser quand même."}
\end{center}
\end{quote} \hfill Gramain, 2024\cite{gramainInterviewAgnesGramain2024}

Enfin, Theodore Turocy revient sur les obstacles rencontrés pour faire reconnaître et financer Gambit : pendant près de 20 ans, le logiciel, pourtant utilisé, ne bénéficie d’aucun soutien institutionnel. Ce manque de reconnaissance et de moyens techniques freine considérablement son développement, malgré son utilité démontrée dans la communauté des théoriciens des jeux. 

Ces témoignages montrent que le travail informatique est souvent invisible, chronophage, et peu valorisé dans les circuits académiques. Les défis rencontrés ne sont pas seulement techniques, mais aussi institutionnels et symboliques, renforçant parfois un sentiment d’isolement ou de sous-valorisation du travail accompli.


%----------------------------------------------------------------------------------------
%	SECTION 2
%----------------------------------------------------------------------------------------
\section{Les différences ou aspects à questionner}

Malgré ces traits communs, plusieurs aspects différencient ces programmeurs-économistes ou méritent d’être questionnés plus précisément.

Premièrement, ces économistes appartiennent à des générations différentes, qui ont connu des stades variés de l’évolution des technologies informatiques. Ces différences générationnelles ont structuré leur accès aux machines, aux langages, et aux pratiques de programmation. 

David Hendry représente la génération des pionniers : il débute sa carrière dans les années 1970, à l’époque des mainframes IBM et des cartes perforées. L’informatique est alors centralisée, peu accessible, et la programmation se fait en Fortran, sur des machines coûteuses et lentes. 

Theodore Turocy, formé dans les années 1990 à Caltech, appartient à une génération qui a grandi avec l’ordinateur personnel à domicile. Il commence à programmer vers 8 ans sur un micro-ordinateur familial, en BASIC, et bénéficie très tôt d’un environnement académique intensément informatisé. À l’université, il combine des cours de sciences appliquées, d’informatique et d’économie, et participe dès ses années de licence à des projets de logiciels (comme Gambit). Il symbolise une intégration précoce et fluide de l’informatique dans la formation des économistes. 

Joshua Epstein, Urs Fischbacher et Richard Pierse se situent dans une position intermédiaire. La formation d’Epstein commence avant la diffusion massive des PC, mais il adopte très tôt les outils computationnels dans ses travaux. Lors du développement de Sugarscape au début des années 1990, aucun environnement de modélisation agent n'existe encore : il doit programmer « à la main », sans interface dédiée. Il appartient à une génération charnière, qui passe de la modélisation mathématique aux simulations informatiques, dans un contexte encore expérimental. De son côté, Fischbacher développe z-Tree au milieu des années 1990, dans un moment où les PC sont déjà largement disponibles, et où l’usage d’ordinateurs en laboratoire devient central dans l’économie expérimentale. Il bénéficie des environnements de développement modernes avec la version orienté objet de Pascal. 

Enfin, Agnès Gramain, est formée dans un environnement proche des grandes structures informatiques universitaires françaises, où l’usage de SAS est encore dominant. Elle témoigne avoir dû faire tourner ses programmes en "batch" pendant la nuit, sur les ordinateurs de l'INSEE. Cependant, dès son doctorat, l’usage de l'ordinateur personnels se développe rapidement. Ainsi, les trajectoires de ces économistes reflètent bien l’évolution des infrastructures informatiques : des mainframes centralisés aux PC personnels, des langages compilés aux environnements interactifs. Chaque génération s’approprie l’informatique dans des conditions techniques, pédagogiques et institutionnelles spécifiques.


Deuxièmement, les parcours des programmeurs-économistes reflètent l’évolution des langages de programmation utilisés en économie au fil des décennies. Si tous ont commencé avec des outils plus complexes à maitrisés ou spécialisés – comme Fortran, C++, GAUSS ou Maple – les standards contemporains semblent désormais converger vers Python et R. En effet, ces langages n'ont pas été conçus en premier lieu pour l'économie. Cependant, leur caractère accessible (syntaxe claire, apprentissage facilité), open source (ils sont accessibles gratuitement, tout le monde peut en consulter l'architecture interne et le fonctionnement) et les écosystèmes puissants qui se sont créés autour d'eux (des communautés d'utilisateurs actifs développent et enrichissent continuellement des bibliothèques spécialisées) leur ont permis de s'imposer comme des outils de référence pour l’analyse statistique, la modélisation économique et le traitement de données. L’histoire des trajectoires de ces économistes est aussi celle de l’évolution de leurs outils de travail. Le langage C++, quoique toujours pertinent, paraît réservé à des applications spécifiques ou à des besoins computationnels particuliers. 

Fischbacher l’a utilisé largement pour créer z-Tree. Utilisé à l’origine par Epstein et d’autres, il tend à être délaissé dans les usages courants, au profit de langages de plus haut niveau et plus lisibles. Sa complexité, ainsi que la courbe d’apprentissage qu’il impose, en font aujourd’hui un outil de niche, réservé à des contextes exigeants. 

Theodore Turocy, utilise aujourd’hui Python dans ses activités de formation et de recherche, et le considère comme un outil adapté aux besoins modernes de la théorie des jeux computationnelle. 

Joshua Epstein, sans mentionner un langage unique, cite également Python parmi les environnements désormais incontournables dans la modélisation agent-based, aux côtés de plateformes comme NetLogo ou Repast. Il souligne combien l’arrivée de ces outils a facilité l’accessibilité de la simulation sociale, par rapport aux débuts plus artisanaux en C++. 

Agnès Gramain, a comme dit plus haut, été formée à SAS à l’ENSAE, le logiciel statistique privilégié par l’INSEE à l’époque. Aujourd’hui, dans le cadre de son enseignement, elle cherche à se tourner vers R, un langage libre, gratuit et plus facilement accessible à ses étudiants. Elle rapporte cependant que cette transition est difficile : les logiques syntaxiques et conceptuelles de R lui semblent très éloignées de celles de SAS, qu’elle maîtrise parfaitement. Ce témoignage met en lumière les obstacles cognitifs et pratiques que peuvent rencontrer les chercheurs confrontés à des évolutions techniques rapides, même lorsqu’ils en saisissent les enjeux pédagogiques. 

En somme, l’évolution des langages utilisés par ces économistes reflète une transition structurelle dans les outils de travail de la discipline : d’environnements spécialisés, souvent payants et complexes, vers des langages plus ouverts, communautaires et pédagogiquement adaptés, comme Python et R. Mais cette transition reste inégalement vécue, selon les trajectoires, les générations et les usages.


Troisièmement, la reconnaissance académique du travail logiciel demeure problématique. C'est un constat partagé largement dans la littérature (Merow et al. 2023\cite{merowBetterIncentivesAre2023}, Howison et al 2011\cite{howisonScientificSoftwareProduction2011}). Tous témoignent de la difficulté à valoriser le travail de programmation dans les meilleurs journaux d'économie. Il est notamment intéressant de constater que Kenneth Judd, l’un des pionniers de l’économie computationnelle, consacre une page de son site internet personnel à une forme de « name and shame » des revues qui, selon lui, ont fait obstacle à l’introduction des méthodes numériques en économie\cite{DocumentationTyrannyTop}. Comme il l’explique en ouverture de cette page :
\begin{quote}
\begin{center}
\textit{``Economics is ignoring the potential of modern computational technologies and related mathematical methods. This website will give examples of the obstacles faced by those who want to bring these tools to economics, and address those who are creating those obstacles.''}
\end{center}
\end{quote} \hfill Kenneth L. Judd, \textit{Documentation for Tyranny of the Top Five}\cite{DocumentationTyrannyTop}

Tous ces programmeurs-économistes ont développé des outils informatiques cruciaux pour leurs recherches – voire pour celles d’une communauté entière – mais la reconnaissance institutionnelle de ces contributions demeure variable, souvent limitée et indirecte. 

Turocy est explicite sur ce point : le logiciel Gambit, auquel il a consacré plusieurs années, est fréquemment absent ou relégué dans les publications finales. Alors même que les résultats sont obtenus à l’aide du logiciel, les articles publiés n’en disent presque rien. Cette invisibilisation du travail computationnel témoigne, selon lui, d’une tension entre l’importance pratique du logiciel et sa faible valeur perçue dans les critères académiques classiques. 

Hendry évoque quant à lui les résistances qu’il a rencontrées dans les années 1980, notamment face à l’idée d’automatiser certaines tâches de l’économétre. Certains collègues voyaient dans les approches computationnelles une menace à l’autonomie intellectuelle, ou une perte des savoir-faire analytiques traditionnels. Malgré le succès de ses méthodes, cette réception initiale montre que le travail logiciel a longtemps été vu comme secondaire ou mécanique, par rapport à la « vraie » théorie. 

Epstein évoque moins directement ce problème, mais souligne l'importance d'établir une épistémologie claire pour valoriser l'approche computationnelle. Son travail sur l’« explication générative » vise précisément à établir cette légitimité. 

Pierse, quant à lui, souligne que la valeur académique d’un outil ne réside pas nécessairement dans sa visibilité directe, mais dans la qualité des résultats empiriques qu’il permet d’obtenir. Ce sont les simulations robustes et les prévisions fiables qui assurent, selon lui, la reconnaissance du travail logiciel – même si celui-ci reste en arrière-plan dans les publications. 

Fischbacher, enfin, a adopté une stratégie différente, plus explicite et volontariste. Lorsqu’il diffuse z-Tree, son logiciel d’expérimentation, il demande systématiquement aux utilisateurs de citer son article méthodologique. Ce modèle de “\textit{\textit{\textit{\textit{citeware}}}}”, qu’il assume pleinement fonctionne bien : l’article de présentation de z-Tree est sa publication la plus citée, avec plus de 13 000 citations en 2025. Fischbacher considère cette méthode non seulement comme un levier de reconnaissance personnelle, mais aussi comme un mécanisme de diffusion efficace du logiciel. 

Ces trajectoires révèlent une ligne de fracture persistante dans la science économique : les contributions informatiques, même essentielles à la production de résultats, peinent à être reconnues comme telles. Qu’il s’agisse d’outils invisibilisés, de résistances disciplinaires ou de stratégies de légitimation par la citation, le travail logiciel reste souvent en tension avec les normes dominantes de valorisation académique.



Un autre point marquant réside dans la diversité des ancrages institutionnels des programmeurs-économistes interrogés. Leurs parcours s’inscrivent dans des espaces professionnels variés, sans qu’un modèle unique ne se dégage. Cette hétérogénéité illustre la plasticité des profils en économie computationnelle, à la croisée de plusieurs champs ainsi que les usages multiples de leurs outils. Cependant, cela reflète aussi une certaine marginalité dans la discipline : beaucoup naviguent entre champs, sans appartenir pleinement à un sous-domaine. 

Richard Pierse mène l’essentiel de sa carrière à la Banque d’Angleterre, où il contribue à l’intégration des outils économétriques dans la modélisation macroéconomique des politiques monétaires. Son travail illustre la place des programmeurs-économistes dans les institutions publiques, au service de la décision économique. 

Joshua Epstein, de son côté, travaille dans des think tanks (Brookings Institution), avant d’enseigner dans des universités de premier plan (Johns Hopkins, NYU). Son parcours est marqué par des passages entre sciences sociales, santé publique et modélisation computationnelle, souvent dans des centres de recherche interdisciplinaire, à la frontière entre politique, épidémiologie et économie. 

Urs Fischbacher construit sa carrière en deux étapes. D’abord en tant qu’ingénieur logiciel dans des entreprises privé et dans des organismes public. Puis, à l’université de Zurich et de Constance, dans des départements de recherche appliquée en économie expérimentale. Il conçoit z-Tree de manière à le rendre utilisable en dehors du milieu académique, notamment dans des contextes de test comportemental, ce qui peu conduire à des usages hors du champ strictement universitaire. 

David Hendry est une figure centrale du département d’économie de l’université d’Oxford, avec une forte influence dans les cercles académiques et dans les institutions de politique économique au Royaume-Uni. Il incarne une position plus classique dans la recherche universitaire, mais avec une attention constante aux applications pratiques de l’économétrie. 

Theodore Turocy alterne entre institutions américaines (Caltech, Northwestern, Texas AM) et britanniques (University of East Anglia). Il occupe actuellement une position hybride entre l’université et l’Alan Turing Institute, où il participe à des projets d’intelligence artificielle fondés sur la théorie des jeux computationnelle. Son profil illustre l’ouverture récente de la recherche académique à des collaborations avec l’informatique avancée et l’IA. 

Agnès Gramain, enfin, mène une carrière académique plus classique, à Dauphine et à l’université de Lorraine, au sein de laboratoires d’économie théorique et appliquée, en mobilisant beaucoup la modélisation micro économétrique. En somme, les programmeurs-économistes interrogés naviguent entre universités, think tanks, institutions publiques et centres de recherche interdisciplinaire. Cette diversité d’ancrages confirme l’absence de “profil type” et reflète la fluidité professionnelle propre à un champ encore en construction, situé à la croisée de plusieurs disciplines et pratiques




Enfin, les trajectoires des programmeurs-économistes révèlent combien le hasard, les opportunités inattendues et le "bricolage" intellectuel jouent un rôle structurant dans la naissance de nombreux projets logiciels. Loin d’être issus d’une planification rationnelle ou linéaire, ces outils sont souvent le fruit de rencontres fortuites, de besoins pratiques ou de circonstances expérimentales. 

Joshua Epstein raconte de manière emblématique la genèse de Sugarscape, conçu avec Robert Axtell lors d’une discussion informelle dans la cafétéria du Brookings Institution. Ils dessinent les premiers éléments du modèle sur des serviettes en papier, dans un esprit d’expérimentation libre, sans cadre théorique préalable ni logiciel existant. Il insiste sur le caractère profondément improvisé de ce moment, soulignant que rien n’était prédéfini : ni la structure du modèle, ni son objectif final. C’est dans ce contexte de création spontanée, encouragé par la liberté offerte par l’institution, que naît un modèle aujourd’hui emblématique de la modélisation agent-based. 

Theodore Turocy souligne également la contingence de son engagement dans le projet Gambit. Il découvre le logiciel lors d’un cours à Caltech, alors qu’il n’était pas initialement destiné à devenir programmeur. C’est un besoin pédagogique – le manque de logiciels adaptés à l’enseignement de la théorie des jeux – qui l’amène à s’impliquer dans son développement. Ce détour imprévu devient progressivement un pilier de sa carrière, même si ce travail reste longtemps sans financement ni reconnaissance académique. 

Urs Fischbacher explique que son entrée dans l’économie expérimentale n’a rien d’un plan de carrière : il travaille d’abord comme ingénieur logiciel, puis rejoint par hasard un laboratoire d’économie qui cherche quelqu’un pour développer un outil informatique d’expérimentation. Il saisit l’opportunité, apprend les exigences du champ, et finit par concevoir z-Tree, devenu un standard international. 

Richard Pierse évoque également des choix de trajectoire façonnés par les circonstances. Ce sont ses emplois d’été au NIESR (National Institute of Economic and Social Research) qui l’initient à la modélisation économétrique appliquée. Ce premier contact, motivé au départ par des considérations alimentaires, l’oriente durablement vers le croisement entre économie et programmation. 

Ces exemples montrent que, bien souvent, les projets logiciels en économie ne naissent ni d’un plan de carrière ni d’un projet de recherche initialement centré sur la programmation. Ils émergent dans les interstices de l’activité scientifique : par nécessité, par chance, ou par expérimentation. Le hasard des rencontres, les contraintes locales, les besoins techniques non satisfaits et les financements opportuns jouent un rôle déterminant dans la structuration de ces trajectoires.\\[2cm]






Après avoir étudié les figures de l’économiste-programmeur, la prochaine étape de ce travail portera sur les logiciels eux-mêmes : leurs usages, leurs architectures, leurs trajectoires de diffusion. Nous nous appuierons notamment sur Renfro, 2004\cite{renfroCompendiumExistingEconometric2004} Il s’agira d’analyser comment ces outils sont conçus, ce qu’ils permettent (ou empêchent), et comment ils structurent les formes de preuve et les régimes d’autorité dans la discipline. À travers cette enquête sur les instruments, c’est toute une histoire technique et intellectuelle de l’économie contemporaine qui se dessine.
Enfin, nous approfondirons la recherche du côté des épistémologies sous-jacentes à chaque famille de méthodes computationnelles, modélisation agent, expérimentation assistée par ordinateur, automatisation des tests économétriques, afin d’en tirer des leçons pour l’avenir de la discipline. Ce détour par les outils et celles et ceux qui les fabriquent n’a rien d’anecdotique : il ouvre sur une réflexion plus large sur les conditions concrètes de production des savoirs économiques aujourd’hui.
