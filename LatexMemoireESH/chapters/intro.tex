
Mary Morgan, dans son livre de 2012, The World in the Model: How Economists Work and Think\cite{morganWorldModel}, décrit l'économie comme une "science outillée", et analyse le rôle centrale des modèles. Un autre outil, moins souvent étudié, est devenu tout aussi structurant.

Il s'agit de l'outil informatique.

Depuis les années 1970, l’usage de l’informatique en économie n’a cessé de s'intensifier. Il s’impose aujourd’hui dans toutes les dimensions de la recherche : simulation, expérimentation, traitement de données ou encore modélisation basée agent. Derrière cette transformation technique, on trouve des trajectoires individuelles bien concrètes : celles des programmeurs-économistes, à la fois chercheurs et développeurs, dont les contributions ne se limitent pas à l’usage d’outils existants, mais incluent la création de logiciels devenus centraux dans certains sous-champs de la discipline (voir Boumans et al. 2023\cite{boumansComputerizationEconomicsThree2023}, Cherrier et al. 2023\cite{cherrierWriteYourModel2023}, Backhouse et al. 2017 \cite{backhouseItsComputersStupid2017}). 

Ce travail s’appuie sur une série d’entretiens réalisés dans le cadre du projet Oral Histories of Economics, avec 7 figures ayant contribué de manière significative à cette hybridation entre économie et informatique : David Hendry\cite{hendryInterviewDavidHendry2024}, Joshua Epstein\cite{epsteinInterviewJoshuaEpstein2024}, Robert Axtell\cite{axtellInterviewRobertAxtell2025}, Theodore Turocy\cite{turocyInterviewTheodoreTurocy2024}, Urs Fischbacher\cite{fischbacherInterviewUrsFischbacher2024}, Richard Pierse\cite{pierseInterviewRichardPierse2024} et Agnès Gramain\cite{gramainInterviewAgnesGramain2024}. À travers leurs récits, il s’agit de documenter, comparer et analyser leurs trajectoires, afin de mieux comprendre comment se forment, s’exercent et se diffusent les compétences de programmation en économie, quels types de profils les incarnent, et comment l'évolution de l'informatique influence les pratiques du programmeur-économiste. 

L’objectif de ce mémoire est double. Il s’agit de dresser un portrait des programmeurs-économistes et des logiciels qu’ils ont conçus, en identifiant à la fois leurs traits communs — modalités d’apprentissage de l’informatique, profils académiques hybrides, diversité des langages mobilisés, rapports de genre, confrontation aux contraintes techniques et institutionnelles — et leurs différences, qu’elles soient générationnelles, institutionnelles ou techniques.

Puis d'analyser ces spécificités. Il s’agira de les mettre en perspective, de les comparer aux évolutions récentes de l’informatique en économie, et d’interroger ce qu’elles révèlent du rôle de la programmation dans la production et la transformation des savoirs économiques.

Le mémoire se déploie trois chapitres. Le premier est consacré à l’analyse biographique des programmeurs-économistes rencontrés, afin de dégager leurs traits communs et leurs différences. Le deuxième s’attache aux logiciels qu’ils ont conçus : leurs objectifs, leurs usages et leurs devenirs. Le troisième propose une analyse de ces trajectoires.

Méthodologiquement, ce travail repose principalement sur l’analyse qualitative des entretiens, qui constituent des sources situées et construites. Comme le rappelle Jullien \cite{jullienInterviewsMethodologicalHistoriographical2018}, les récits oraux sont traversés par des enjeux de légitimité et par des choix narratifs. Ils doivent donc être lus avec précaution, en tenant compte des silences, des oublis et des stratégies discursives. Nous avons cherché à croiser ces témoignages avec d’autres sources (documents publics, littérature secondaire) afin d'en proposer une lecture réflexive et critique.

